\documentclass[10pt,a4paper]{article}

\usepackage{graphicx}
\usepackage{amssymb}
\usepackage{amsthm}
\usepackage[cm]{fullpage}
\usepackage{xcolor}
\usepackage[parfill]{parskip}
\usepackage{verbatim}
\usepackage{hyperref}

%%%
\usepackage{listings}
\lstset{
  basicstyle=\ttfamily,
  breaklines=true,
  frame=none,
  keepspaces=true,
  numberstyle=\small\color{black!80},
  xleftmargin=.0in,
  language=Prolog,
  deletekeywords={clause,not,var},
  commentstyle=\color{black!40},
  escapechar=|,
  aboveskip=\medskipamount,
  belowskip=\medskipamount,
  backgroundcolor=\color{black!20},
}
\newcommand{\inlinecode}[1]{\texttt{\small #1}}
%%%

\title{Homework assignment 2 -- Symbolic Systems I -- UvA, June 2020}
\author{\it Your Name Here}
\date{}

\begin{document}
\maketitle

\section*{Question 1}

Provide your answer to question 1 here.

Propositional logic formulas are typeset nicely as follows, for example:
\[ \begin{array}{r l}
  \varphi = &
    (x_1 \vee \neg x_2 \vee x_3)
    \wedge
    (x_2 \vee \neg x_1 \vee x_4)
    \wedge
    (x_3 \vee \neg x_4 \vee x_5)
    \wedge {} \\
  & (x_2 \vee x_3 \vee \neg x_4)
    \wedge
    (x_3 \vee \neg x_4 \vee \neg x_5)
    \wedge
    (x_4 \vee \neg x_5 \vee x_6) \\
\end{array} \]

\section*{Question 2}

Provide your answer to question 2 here.

If you want to include code (e.g., answer set programs), you can do it nicely as follows:

\begin{lstlisting}
a :- b, not c.
b :- not c.
\end{lstlisting}

\section*{Question 3}

Provide your answer to question 3 here.

If you want to use DL notation,
you can do that using, e.g., \verb|\sqsubseteq|, \verb|\sqcup|, etc%
\footnote{See, e.g., \url{https://oeis.org/wiki/List_of_LaTeX_mathematical_symbols\#Relation_operators}.}:
$C \sqsubseteq \neg (C \sqcap (D \sqcup E))$.

\section*{Question 4}

Provide your answer to question 4 here.

\section*{Question 5}

Provide your answer to question 5 here.


\end{document}  